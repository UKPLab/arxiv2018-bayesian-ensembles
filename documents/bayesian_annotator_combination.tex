% Root file for the contributions of a Decision Making with Multiple Imperfect
% Decision Makers
%%%%%%%%%%%%%%%%%%%%%%%%%%%%% Springer %%%%%%%%%%%%%%%%%%%%%%%%%%
\documentclass[conference]{IEEEtran}
%\documentclass{sig-alternate-05-2015}
%\documentclass[a4paper]{article}
% \documentclass[12pt,a4paper]{article}

%\usepackage[top=2cm, bottom=2cm, left=2cm, right=2cm]{geometry}
%\usepackage[hmargin=2.9cm,vmargin=2.9cm]{geometry}

%\usepackage{times}
\usepackage{graphicx}
%\usepackage{tabularx}
%\usepackage[toc]{appendix}
%\usepackage{epsfig}

\usepackage[fleqn]{amsmath}
\usepackage{amssymb}
\usepackage{amstext}
\usepackage{amsfonts}
\usepackage{amsthm}

\usepackage{cite}
\usepackage{algorithm2e}
\usepackage{array}
 \usepackage[caption=false,font=footnotesize]{subfig}
% \usepackage{fixltx2e}
 %\usepackage{stfloats}
 \usepackage{url}

\newcommand{\bs}{\boldsymbol}  
\newcommand{\wrtd}{\mathrm{d}}

%\usepackage{booktabs}

\makeatletter
\makeatother %some sort of hack related to the symbol @

% \usepackage{breakcites}
% \usepackage{etoolbox}
% \patchcmd{\@citex}{,}{;}{}{}

\DeclareMathOperator*{\argmax}{\arg\!\max\!} %argmax operator


%%%%%%%%%%%%%%%%%%%%%%%%%%%%%%%%%%%%%%%%%%%%%%%%%%%%%%%%%%%%%%%%%

\title{ 
%A Bayesian Method for Vetting and Combining Crowds of Text Annotators
A Bayesian Method for Combining Multiple Unreliable Text Annotators
}

\author{\IEEEauthorblockN{Anonymous}
\IEEEauthorblockA{Anonymous, \\
Anonymous \\
Email: anonymous}
\and
\IEEEauthorblockN{Anonymous}
\IEEEauthorblockA{Anonymous, \\
Anonymous \\
Email: anonymous}
\and
\IEEEauthorblockN{Anonymous}
\IEEEauthorblockA{Anonymous, \\
Anonymous \\
Email: anonymous}
}

\begin{document}

\maketitle

\begin{abstract}
A common task in NLP is sequence labelling, which is performed by both human annotators to 
produce training data and by automatic classifiers that extract information from text.
However, different annotators can often disagree and may have highly varying levels of reliability,
particularly when crowdsourcing is used to annotate spans of text. 
High error rates can be mitigated by combining annotations from multiple annotators,
a technique that is also used by ensembles of classifiers to boost performance.
Existing approaches that model the biases and error rates of annotators have been shown to 
improve over simple heuristics such as majority voting. However, existing methods
ignore the sequential nature of text span annotations and may therefore underperform.
We propose a new Bayesian technique to combined multiple annotators of differing reliability 
and make the software available publicly. 
Using a series of simulations, we show how several different probabilistic
and heuristic approaches perform under different conditions. 
We illustrate how our approach can improve sequential classification performance on a 
real-world argumentation mining task by using it to combine both human annotators and 
an ensemble of automated classifiers.
\end{abstract}

% For peer review papers, you can put extra information on the cover
% page as needed:
% \ifCLASSOPTIONpeerreview
% \begin{center} \bfseries EDICS Category: 3-BBND \end{center}
% \fi
%
% For peerreview papers, this IEEEtran command inserts a page break and
% creates the second title. It will be ignored for other modes.
\IEEEpeerreviewmaketitle

%%%%%%%%%%%%%%%%%%%%%%%%%%%%%%%%%%%%%%%%%%%%%%%%%%%%%%%%%%%%%%%%%

\section{Introduction}\label{sec:intro}

Scientific research relies on humans to recognise important patterns in data – even if we employ automated methods, these typically require training labels produced by human annotators. 
Natural language processing (NLP) often requires people to annotate segments of text, which we then use to train machine learning algorithms and evaluate our results.
Many NLP tasks require training data in the form of annotations of phrases and propositions in text. These annotations are spans of varying length, and different pieces of text may contain different numbers of spans. An example is highlighting claims in argumentative text. Annotators will typically make mistakes and may disagree with each other about the correct annotation, even if they are experts. When processing large datasets we may use crowdsourcing to reduce costs/time of experts, which increases the amount of noise and disagreements as the annotators are non-experts. Therefore, we require a method for aggregating text span annotations from multiple annotators.

Heuristic rules could be applied, such as taking intersections of annotations, or majority labels for individual words to determine whether they form part of a span or not. However, this does not account for differing reliability between workers (e.g. there may be spammers, people who do not understand the task) and the theoretical justification for these rules is often unclear. Therefore it may not be possible to apply simple heuristics to obtain gold-standard labels from a crowd. 

In this paper develop a Bayesian machine learning algorithm for combining multiple unreliable text annotations.
The method we propose is based on the classifier combination method described by \cite{kim2012bayesian}, 
which was shown to be effective for handling the unreliable classifications provided by a crowd of workers. A scalable implementation of this method using variational Bayes was described by \cite{simpsonlong}, which we use as the basis for our implementation in the current work. This paper provides the following contributions:
\begin{itemize}
  \item Propose a probabilistic model for combining classifications to combine annotations over sequences of words
  \item Describes and tests a scalable inference algorithm for the proposed model that adapts the existing variational Bayes implementation for classifier combination
  \item Compares the approach on real-world NLP datasets with simple heuristic methods (e.g. mode) and alternatives such as weighted combinations
  \item Demonstrates how using the proposed Bayesian model enables an active learning approach that improves crowdsourcing efficiency
\end{itemize}

\subsection{Notes on Applications and Datasets}

There are several annotation tasks for NLP that we are interested in:
\begin{itemize}
  \item Argument component labelling -- identifying claims and premises that form an argument. This requires marking individual sentences, clauses, or spans that cross sentence boundaries. Some schemas allow for the component to be split so that it consists of multiple spans with excluded text between the spans.
  \item Semantic role labelling (SRL).
\end{itemize}



\section{Modelling Text Span Annotations}\label{sec:model}

We model annotations using the IOB schema, in which each token in a document is labelled as either I (in), O (out), or B (begin). The IOB schema requires that the label I cannot directly follow a label O, since a B token must precede the first I in any span. The IOB schema allows us to identify whether a token forms part of an annotation or not, and the use of the B label enables us to separate annotations when one annotation span begins immediately after another without any gap. This schema does not permit overlapping annotations, which are typically undesirable in crowdsourcing tasks where the crowd is instructed to provide one type of annotation. The schema also does not consider different types of annotation, although it is trivial to extend both the schema and our model to permit this case. Using a single model for different types of annotation may be desirable if the annotators are likely to have consistent confusion patterns betweeen different annotation types. 

We propose an extension of the independent Bayesian classifier combination (IBCC) 
model~\cite{kim2012bayesian} for combining annotations provided by a crowd of unreliable annotators. We refer to our model as Bayesian annotator combination or BAC. In BAC, we model the text annotation task as a sequential classification problem, where the true class, $t_i$, of token $i$ may be I, O, or B, and is dependent on the class of the previous token, $t_{i-1}$. This dependency is modelled by a transition matrix, $A$, as used in a hidden markov model. Rows of the transition matrix correspond to the class of the previous token, $t_{i-1}$, while columns correspond to values of $t_i$. Each row is therefore a categorical distribution. 

We model the annotators using a confusion matrix similar to that used in \cite{simpsonlong}, which captures the likelihood that annotator $k$ labels token $i$ with class $c_i^{(k)}$, given the true class label, $t_i$, and the previous annotation from $k$, $c_{i-1}^{(k)}$. The dependency between $c_i^{(k)}$ and $t_i$ allows us to infer the ground truth from noisy or biased crowdsourced annotations. There is also a dependency on the previous worker annotation, since these are constrained in a similar way to the true labels, i.e. the class I cannot follow immediately from class O. Furthermore, mistakes in the class labels are likely to be correlated across several neighbouring tokens, since annotations cover continuous
spans of text. The confusion matrix, $\bs\pi^{(k)}$, is therefore expanded in our model to a three dimensional transition-confusion mtrax, where the element $\pi_{j,l,m}^{(k)} = p(c_i^{(k)} = m | c_{i-1}^{(k)}=l, t_i=j)$. Within $\bs\pi^{(k)}$, the vector $\bs\pi_{j,l}^{(k)} = \{ \pi_{j,l,1}^{(k)},...,\pi_{j,l,L}^{(k)}\} $, where $L$ is the number of class labels, represents a categorical distribution over the worker's annotations conditioned on the ground truth and their previous annotation.

\subsection{Generative Model}

In the BAC approach, the model described above is given a Bayesian treatment by placing prior distributions over the state transition matrix $A$ and worker confusion matrices $\bs\pi^{(k)}$. The generative process is as follows. 

\textbf{Ground truth:} For each class label $j=\{I, O, B\}$, we draw a row of the transition matrix, $A_j \sim \mathrm{Dir}(\bs\beta_j)$, where $\mathrm{Dir}$ is the Dirichlet distribution. 
%We also draw an initial state distribution, $\bs\kappa=\{\kappa_1, ..., \kappa_L\} \sim \mathrm{Dir}(\bs
%\nu)$, where $\kappa_j=p(t_1=j)$. 
For each document $i$ in a set of $N$ documents, we now draw a sequence of class labels $\bs t_i = [t_{i,1}, ..., t_{i, T_i}]$ of length $T_i$. For $\tau=1$, we draw the first label in each sequence from 
$t_{i,\tau} \sim \mathrm{Categorical}(\bs A_{O})$, 
%$t_{i,\tau} \sim \mathrm{Categorical}(\bs \kappa)$, 
then for $\tau > 1$, we draw subsequent labels from $t_{i,\tau} \sim \mathrm{Categorical}(\bs A_{t_{i,\tau-1}})$. The first label in each sequence uses hyperparameters $\bs A_{O}$ because there is no previous annotation, so we assume that the state $t_{i,0}$ prior to the document start is not part of an annotation, and therefore $t_{i,0}=O$ is an outside or $O$ token. 

\textbf{Worker annotations:} For each worker $k\in\{1,...,K\}$, true label $j\in\{1,...,L\}$, and previous worker label $l=\{1,...,L\}$, we draw vectors $\bs\pi_{j,l}^{(k)} \sim \mathrm{Dir}(\bs\alpha^{(k)}_{j,l})$, which make up the three-dimensional transition-confusion matrix. We now draw annotations for each worker $k$ for each document $i$, starting with the first term, $c_{i,1}^{(k)} \sim \mathrm{Categorical}( \bs\alpha^{(k)}_{t_{i,1}, O} )$, then subsequent terms $c_{i,\tau}^{(k)} \sim \mathrm{Categorical}( \bs\alpha^{(k)}_{t_{i,\tau}, c_{i,\tau-1}^{(k)}} )$. As with the true labels, the first annotation in each sequence uses hyperparameters $\bs\alpha^{(k)}_{t_{i,1}, O}$ because we assume that the annotation prior to token $1$ is equivalent to an $O$ annotation. 

\subsection{Variational Bayes (VB) Algorithm}

We modify the mean-field variational Bayes algorithm proposed by \cite{simpsonlong}, 
which assumes an approximate posterior distribution that factorises between the parameters and 
latent variables. For our proposed model, the variational approximation is given by:
\begin{equation}
  q(\bs t, \bs A, \bs\pi^{(1)},...,\bs\pi^{(K)}) = q(\bs t)\prod_{j=1}^L \left\{ q(\bs A_j)\prod_{l=1}^L \prod_{k=1}^K  q(\bs\pi_{j,l}^{(k)}) \right\}
\end{equation}
Below, we summarise the algorithm used to optimise this distribution to obtain an approximate posterior.
We then define the variational factors and expectation terms needed to perform each step of the algorithm. The procedure is as follows:
\begin{enumerate}
 \item \label{step:1} Initialise variational factors for parameters
$\bs A_j$, $\forall j$ and
$\bs\pi_{j,l}^{(k)}, \forall j, \forall l, \forall k$, e.g. by setting to prior distributions.
 \item \label{step:2} Calculate $\mathbb{E}\left[\log \bs A \right]$ 
and $\mathbb{E}\left[\log\bs\pi^{(k)} \right], \forall k$ given the current factors 
$q(\bs A_j)$ and $q(\bs\pi_j^{(k)})$.
 \item Update the variational factor for the ground truth labels, $q(\bs t)$, given 
the expectations $\mathbb{E}\left[\log\bs\pi^{(k)} \right], \forall k$, and
$\mathbb{E}\left[\log \bs A \right]$, using the forward-backward
algorithm\cite{ghahramani2001introduction}, which will be explained further below.
 \item Update the variational factors $q(\bs\pi_j^{(s)}), \forall j, \forall s$ for the confusion matrices given current estimate for $q(\bs t)$.
 \item \label{step:4} Update the variational factor for the transition matrix rows
$q(\bs A_j), \forall j$ given the current estimate for $q(\bs t)$.
 \item \label{step:6} Check for convergence in the ground truth label predictions,
$\mathbb{E}\left[\bs t\right]$, or in the variational lower bound. 
The latter may be more expensive to compute but gives stronger guarantees of convergence. 
If not converged, repeat from step \ref{step:2}.
\item \label{step:7} Output the predictions for the true labels, $\mathbb{E}\left[\bs t\right]$ given the converged estimates of the variational factors.
\end{enumerate}

\subsection{Variational Factors}

Steps 1 to 5 of the algorithm above initialise then iteratively update each variational factor in turn. Each update increases the lower bound on the model evidence by optimising one variational factor given the current estimates of the others. In this section we first provide equations for computing the variational factors, then show how to initialise them and compute the expectation terms required by the algorithm.

For the true labels, $\bs t$, the optimal variational factor is:
%\log\pi^{(k)}_{t_{i,\tau},c^{(k)}_{i,\tau-1},c^{(k)}_{i,\tau}}
\begin{align}
  \log q^*(\bs t) = \mathbb{E}_{q} \left[ \sum_{i=1}^N \sum_{\tau=1}^{T_i} \bigg\{ \log p(t_{i,\tau} | t_{i,\tau-1}, \bs A ) \right. \nonumber \\
  \left. + \sum_{k=1}^K p(c_{i,\tau}^{(k)} | t_{i,\tau}, c_{i,\tau-1}^{(k)}, \bs\pi^{(k)})
  \bigg\} \right] + \mathrm{const}, \nonumber\\
   \label{eq:qstar_t}
   =  \sum_{i=1}^N \sum_{\tau=1}^{T_i} q^*(t_{i,\tau}), \\
 %\log p(t_{i,\tau} | t_{i,\tau-1}, \bs A, \bs c) 
 q^*(t_{i,\tau}) = \sum_{j=1}^L \mathbb{E}_q[p(t_{i,\tau-1}=j | \bs c_{i,1:\tau})] 
 \log A_{j,t_{i,\tau}} \nonumber\\ 
 %\sum_{j'=1}^L p(t_{i,\tau}=j') % \sum_{i=1}^N \sum_{\tau=1}^{T_i}
   \sum_{k=1}^K \log\pi^{(k)}_{t_{i,\tau},c^{(k)}_{i,\tau-1},c^{(k)}_{i,\tau}}
   +  p(\bs c_{i,\tau+1:T_i} | t_{i,\tau}) + \mathrm{const}, \nonumber\\
\end{align}
where...

. The dependency on the previous true label and the subsequent label can be solved using the forward-backward algorithm.  

For convenience, we define the variational posterior for each token as $r_{i,\tau,j} = \mathbb{E}_q[p(t_{i,\tau}=j | \bs c)]$.

The updated variational factor for each vector in the three-dimensional worker transition-confusion matrices is:
\begin{align}
  \log q^*(\bs\pi_{j,l}^{(k)}) = \sum_{m=1}^J N_{j,l,m}^{(k)}\log\pi_{j,l,m}^{(k)} + \log p(\bs\pi_{j,l}^{(k)} | \alpha_{j,l}^{(k)}) + \mathrm{const},
\end{align}
where $N^{(k)}_{j,l,m} = \sum_{i=1}^N\sum_{\tau=1}^{T_i} r_{i,\tau,j} \delta_{m,c^{(k)}_{i,\tau}}$ are
pseudo-counts and $\delta$ is the Kronecker delta. Since we assumed Dirichlet priors, the variational 
factor is also a Dirichlet distribution with parameters $\bs a_{j,l}^{(k)} = \bs\alpha_{j,l}^{(k)} + \bs N_{j}^{(k)}$, where $\bs N_j^{(k)}=\left\{ N_{j,l,m}^{(k)}, \forall m \right\}$. 

Equation \ref{eq:qstar_t} makes use of a three-dimensional matrix term $\mathbb{E}[\log \pi^{(k)}]$, which is computed in step 2
of the VB algorithm summarised above. Each element of this matrix is computed using the following:
\begin{align}
  \mathbb{E}[\log \pi_{j,l,m}^{(k)}] = \Psi(a^{(k)}_{j,l,m}) - \Psi(\sum_{m=1}^L a^{(k)}_{j,l} ),
\end{align}
where $\Psi$ is the digamma function.

The transition matrix for the ground truth labels has the following variational factors for each of its rows:
\begin{align}
  \log q^*(\bs A_{j}) = \sum_{i=1}^N\sum_{\tau=1}^{T_i} r_{i,\tau-1,j} \sum_{j'=1}^L r_{i,\tau,j'}\log\bs A_{j,j'} \nonumber\\
  + \log p(\bs A_j | \bs\beta_j) + \mathrm{const} \nonumber\\
  = \sum_{j'=1}^L N_{j,j'}\log\bs A_{j,j'} 
  + \log p(\bs A_j | \bs\beta_j) + \mathrm{const},
\end{align}
where $N_{j,j'} = \sum_{i=1}^N \sum_{\tau=1}^{T_i} r_{i,\tau-1,j}r_{i,\tau,j'}$ are pseudo-counts of the 
number of times that class $j$ follows class $j'$. The variational factor for $\bs A_j$ is Dirichlet distribution with parameters $\bs b_j = \bs\beta_j + \bs N_{j}$, where $\bs N_{j} = \left\{ N_{j,j'} , \forall j' \right\}$.

For Equation \ref{eq:qstar_t} we also require a term $\mathbb{E}[\log A]$ that is computed in step 2
of the VB algorithm. Using the variational Dirichlet distribution over $\bs A$, we can compute each element as:
\begin{align}
  \mathbb{E}[\log A_{j,j'}] = \Psi(b_{j,j'}) - \Psi(\sum_{j'=1}^L b_{j,j'} ).
\end{align}


\section{Alternative Methods}\label{sec:alt}

To date, a number of methods have been used to reduce annotations from multiple workers to a single gold-standard set. These approaches make use of both heuristic and statistical techniques. This section outlines commonly-used baselines and state-of-the-art methods that we later compare against our method.

\subsection{Majority/Plurality Voting}

For classifications, a simple heuristic is to take the majority label, or for multi-class problems, the most popular label. Examples for NLP classification problems include sentiment analysis\cite{sayeed2011crowdsourcing},.... With text spans, we can use the IOB classes and choose the most popular label for each word, but there are a number of cases where the resulting spans would not follow the constraints of the schema, and an additional step is required to resolve these issues. The problems occur when annotators disagree about the starting and ending points of an annotation:
\begin{itemize}
  \item The votes for a token being inside a span can be split between the classes I and B, which could lead to tokens being excluded from spans even when most have marked them as inside. 
  \item The voting process can lead to spans of I tokens with no preceding B token if there is only a minority of annotators who marked did not agree on the first token. 
  \item The spans from different annotators could partly overlap, causing the overlap area itself to be marked as a separate span. In some cases, this may be a valid annotation, while in others it would be obvious to anyone reviewing the annotation that it is an artefact of the aggregation method. There does not seem to be a simple fix here, except for requesting more annotations from other workers. With a sufficient number of annotations, we expect the problem to be resolved.
\end{itemize}
In our experiments, we define a baseline \emph{majority voting} method, which addresses the problems described above as follows. We resolve the first problem using a two-stage voting process. First, we combine the I and B votes and determine whether each token should be labelled as O or not. Then, for each token marked as I or B, we and perform another voting step to determine the correct label. This resolves cases where annotators disagree about whether a span should be split into two annotations. To resolve the second problem of aggregated spans without a B token at the start, we mark the first I token in any aggregate span as B.  

The voting procedure outlined above produces annotations where the annotations of at least 50\% of workers intersect. A stricter approach can be used, which requires that all the annotators mark a token for it to be included (e.g. \cite{farra2015annotating}). We refer to this approach as the \emph{intersect} method. For tasks where workers are likely to miss many spans, it is also possible to lower the threshold so that we do not require a majority of workers to mark a token as I/B before we accept it as such during aggregation.

\subsection{Worker Accuracy-Based Methods}

Determine worker accuracy from a number of gold-standard tasks. Weight the workers' votes by accuracy and apply the majority voting approach above to produce a \emph{weighted majority voting} method.

An interesting approach is used by \cite{hsueh2009data} that takes into account amibiguity in sentiment classifications. It is unclear whether this can be generalised to other types of annotation such as argument components. 

The weights can also be obtained using unsupervised and semi-supervised learning. In this case we use an EM algorithm, in which we initialise the true annotations using the majority voting method, then use these to compute worker accuracies. The true annotations are then re-estimated using a weighted majority vote. The process repeats until convergence. This method is labelled \emph{weighted majority voting (EM)}. 

\subsection{Clustering Methods}

Cluster the annotations, e.g. using a mixture model with annotation centre and spread, or by merging the boundaries somehow. See Zooniverse annotation work -- could discretize this?

\subsection{Other Solutions}

The level of disagreement in annotations for a particular piece of text can be used to determine whether an annotation is of a insufficient quality to keep (e.g. \cite{sayeed2011crowdsourcing,hsueh2009data}. This can be achieved using the majority voting method, but adjusting the threshold for classifying a token as I/B from 50\% to something higher. 

\emph{Human resolution}: an additional worker selects the correct answer from the annotations provided by the initial set of workers, e.g. \cite{dagan2016specifying}. To reduce costs, the human resolution step could be applied only to text with large amounts of disagreement.


\section{Experiments with Synthetic Data}

%
%\subsection{Synthetic Data}
%
%We use synthetic data to illustrate the strengths and weaknesses of different methods by varying one independent
%dataset and tracking the performance metrics of each method:
%\begin{enumerate}
%  \item When do other methods outperform simple majority voting? Show performance against worker accuracy.
%  Similar experiments were carried out with a different set of baselines in \cite{Simpson2014Thesis}, Section 2.5.2, with all workers having similar accuracies, and Section 2.5.4, where some workers are noisy and others are highly accurate. Here, we vary the average accuracy of workers, with lower average accuracy leading to more diversity between workers. Analyse whether the full IBCC confusion matrix offers benefits over MACE due to worker accuracy varying between classes when accuracy is lower. 
%  \item How well do MACE, IBCC, and seq-BCC handle worker bias? Show performance against worker bias toward one class. This is a comparable experiment to \cite{Simpson2014Thesis}, Section 2.5.3, which notably does not include MACE.
%  \item How well does each method handle data sparsity? (a) Vary the amount of observations per worker. Expect MACE to perform well, seq-BCC may suffer from the larger confusion matrix. (b) Vary the number of observations per data point. 
%  \item Do MACE and IBCC still work with unbalanced datasets? Test with unbalanced class distributions, i.e. starting from $p(B) = p(I) = p(O) = .3$, decrease
%  $p(B)$ and $p(I)$ until $p(O) = .99$.
%\end{enumerate}
%In each case, we use a set of default values for the variables that are not currently being tested. These are
% chosen so that all methods perform well (e.g. 90\% accuracy) under the best conditions of the test. E.g. for the test where we vary worker accuracy, we set bias and sparsity to be low so that the performance of all methods is good when workers are 80\% accurate.
%
%For all methods except seq-BCC, we compute performance metrics both before and after the valid annotation post-processing step, which is required to ensure that I tokens do not follow O tokens.
%
%Performance metrics:
%\begin{enumerate}
%  \item Metrics that evaluate the quality of the most probable class labels: recall, precision and F1-score (of B class and I class separately), accuracy (mean over classes)
%  \item Metrics that evaluate the confidence values output by the models: area under ROC curve or AUC (separate for B and I classes), and cross entropy error %or brier score (mean over classes)
%  \item Annotation count error: the mean difference between the number of annotations produced by the model and the true number.
%  \item Number of invalid labels that must be corrected by post-processing.
%  \item Mean length of annotations compared to the ground truth: show whether some methods find annotation fragments.
%\end{enumerate}
%
%We also evaluate the competence scores estimated by each method. First, we compute the ground truth from the 
%synthetic confusion matrices. We use a weighted average over classes (weighted by class frequency) to produce the 
%overall worker accuracy. Then, we compute accuracy from seq-BCC, MACE and IBCC using similar methods. 
%For each method, we can compute the mean and STD of cross entropy error between the estimated and the ground truth
% confusion matrices. We then reproduce the plots described above, but showing the cross entropy error for the 
% competence estimates. 

%\subsection{Real-world Data}
%
%We investigate the performance on some real datasets to show how well the methods work when combining real workers. Besides the performance metrics mentioned above, we also quantitatively analyse examples of where seq-BCC outperforms other methods to show where they may have trouble forming valid or grammatically correct annotations, e.g. where the starting token is incorrect. 
%Ideally, we would also analyse whether the annotations produced are grammatically sensible, e.g. if there is conflict between workers about where an annotation should start, the method should choose one valid start point, not an invalid start point that lies in between. These points can be evaluated by computing:
%\begin{enumerate}
%\item The percentage of "sensible" annotations, as judged by an expert for the task. For tasks where this data is not available, we can use the following metrics to gauge annotation quality.
%\item The percentage of each method's annotations that have an exact match in the gold standard (exact annotation precision)
%\item The percentage of gold standard annotations that have an exact match in the method's output (exact annotation recall)
%\item Mean and variance of no. tokens difference to nearest annotation; this is averaged over the annotations, rather than the number of tokens, so gives a greater indication of how well they matched the gold standard.
%\end{enumerate}

We run several method comparisons using two NLP datasets to test whether the quality of aggregated labels is 
improved by (a) the more sophisticated worker models described in Section \ref{sec:model},
(b) the inclusion of text features into the graphical model or (c) a Bayesian approach. 
We further test whether Bayesian approach facilitates more efficient active learning of sequential annotations from crowds and whether integrating the LSTM into the ensemble of annotators improves performance further.
Our experiments consist of three tasks: (1) aggregating crowdsourced labels, (2) training the LSTM sequence tagger of Lample et al. ~\shortcite{lample2016} using aggregated labels, and (3) actively selecting batches of documents for crowdsourced annotation.
 
\section{Experiments with Real Data}
 
\section{Datasets}

We use two datasets containing both crowdsourced sequential annotations and gold annotations. 
The \emph{NER} dataset contains $1,393$ English documents from the CoNLL 2003 named-entity recognition dataset
~\cite{conll_ner_2003}, all of which contain gold labels for four named entity categories (PER, LOC, ORG, MISC). Of these, we use crowdsourced labels provided by \cite{rodrigues2014} for $415$ documents.
We also test on the \emph{PICO} dataset, introduced by Nguyen et al. ~\cite{nguyen2017aggregating},
containing $4,740$ medical paper abstracts, all of which have been 
annotated by a crowd to indicate text spans that identify the population enrolled in a clinical trial. There are gold labels for $191$ documents.

\section{Evaluation metrics}

For NER we use the established CoNLL 2003 F1-score, which is computed at the level of annotated spans that must match exactly to be considered correct. This measure is intuitive because complete named entities must be marked to be of value. For PICO, we use the relaxed F1-measure defined in ~\cite{nguyen2017aggregating}, which counts the matching fractions of spans when computing precision and recall. 
%We additionally compute the root mean squared error in the span lengths, i.e. the difference between the  % actually it's not quite that. We computed the difference in mean span lengths. This is already captured by F1 score for pico and better described by the span-level-precision and recall. Our metric might be more if we didn't take the absolute so we could see if spans were often too long or too short.
To evaluate the probabilities produced by each aggregation method, which may be useful for decision-making tasks such as active learning, we also compute the cross entropy error.

\section{Evaluated methods}

As well-established non-sequential baselines, we include token-level majority voting (\emph{MV}), \emph{MACE} \cite{hovy2013learning}, Dawid-Skene (\emph{DS}) ~\cite{dawid_maximum_1979}.  We also test independent Bayesian classifier combination (\emph{IBCC}) \cite{kim2012bayesian,simpson_long_paper}, which can be seen as a Bayesian treatment of Dawid-Skene. 

Next, we test the sequential \emph{HMM-Crowd} method \cite{nguyen2017aggregating}. This method uses a mixture of maximum \emph{a posteriori} (or smoothed maximum likelihood) estimates for the worker model, and variational inference for the transition matrix and feature model. The worker model uses a simplification of the DS confusion matrix that models only the probability that a worker labels correctly given each true label class. HMM-Crowd is the current state-of-the-art and allows us to compare our approach against a model without a fully Bayesian treatment. 

We test our proposed method, Bayesian sequence classifier combination (BSCC) in several configurations. Firstly, with different worker models:
\begin{enumerate}
\item accuracy model (\emph{BSCC-acc}): each worker is represented by a single parameter encoding $p( correct label )$
\item Spammer model (\emph{BSCC-MACE}): proposed for MACE \cite{hovy2013learning}, each worker has a parameter encoding 
$p( worker is spamming )$ and a set of $L$ parameters encoding $p( label | worker is spamming)$.
\item Confusion matrix (\emph{BSCC-IBCC}): as in \cite{simpsonlong}, each worker has a matrix of parameters containing $p( label l | true class j)$.
\item sequential confusion matrix (\emph{BSCC-seq}): as described in Section \ref{sec:model}, extends the confusion matrix using an HMM to model label transitions.
\end{enumerate}
Secondly, with different feature models:
\begin{enumerate}
\item No text features (\emph{NF}): only the crowdsourced labels are taken into account when labelling each token. Has the advantage of being task-independent and hence may be more suitable for cases where individual words are uninformative.
\item Independent text features (\emph{IF}): the probability of a token is independent of the sequence conditioned on the true label of the token. This is a standard emission model for an HMM.
\item Integrated LSTM (\emph{intLSTM}): the LSTM is integrated into the variational inference loop as described in Section \ref{sec:model}.
\end{enumerate}


\section{Aggregating Crowdsourced Labels}

In this task, we use the aggregation methods to combine crowdsourced labels and evaluate their outputs against the gold standard.
For NER, we split the $415$ crowd-labelled documents into 50\% validation and test sets as in Nguyen et al. ~\shortcite{nguyen2017}. We run the methods on crowd labels from all $415$ documents, then evaluate on either the
validation or test set.
For PICO, we also split the gold-labelled documents randomly into 50\% validation and test sets. However, in this case, we run the methods on all $4,740$ crowd-labelled documents. The results for this dataset are not directly comparable
with those of Nguyen et al. ~\shortcite{nguyen2014aggregating}, since their test and train splits were not available 
and they appear to have used a subset of the publicly-available dataset with on average 5 annotators per documents, rather than the 6 per document in the complete dataset.

% HMM Crowd -- report their method's results without bugfixes. Cannot say the improvement is purely down to Bayesian treatment though. Let's do this for now, then we can add the more direct comparison to a revised draft if accepted. OR include both, and label the modified HMM_Crowd differently to highlight our contribution here. But we lack the un-revised version for PICO -- look for older runs to see if we have it?

%Sentence separation -- why does it improve performance if we don't split docs into sentences?

% The discrepancy between the latest results and some previous ones, e.g. DS, BSCC-seq-IF with 
% same priors, BSCC-Vec-IF for NER? For BSCC, it may be due to different conf mat for the IF data model.
% For DS, I think it was still using the priors chosen for IBCC, so results changed when nu0 changed.

% TODO: re-run NER task 2 with BSCC-seq. -- running
% TODO: run BSCC-seq, DS and MACE with pico task 1. -- running
% TODO: still have an unexplained difference between HMM-crowd and BSCC-Vec. Is it the initialisation? The attempt to test the same hyperparameter values as HMM-crowd failed -- why is that? Diff may also be due to the use of confusion matrix for the data model - check that is working okay.

Note that the token-level F1-score can be skewed upwards by matching a few long spans correctly, but is useful for PICO because it shows up cases where the spans matched but the predictions were split, i.e. B is used instead of I. With non-strict entity matching, the precision and recall can be 100\% even though the prediction is split into multiple spans.
Token-level F1-score catches this because it penalises the erroneous B tokens. With strict entity-level F1-score, the matches must be exact, so split spans would receive no credit.

% ! means we need to replace the prec, rec and entity f1 with the new versions
% ...however, this means the scores have now dropped below nguyen's, except DS...
% * means we have values from the old dataset

% with the new f1 span metric, precision is now lower for BSCC than for IBCC.
% Precision was higher then for IBCC before, which means that more spans identified by BSCC are covered by the ground truth than for IBCC. However, prec has dropped because BSCC spans are too short. 
% Recall also dropped below that of IBCC for BSCC-IBCC. This suggests some predicted spans are too long. 
% The difference in the models is the ground truth HMM and the priors. 
% The HMM could cause the drop in precision if spans are not sticky enough, i.e. if I-O 
% transition is more likely than I-I. However, this should not be stronger than the p(O) used by IBCC.
% Alternatively, there may be an error where labels switch between types caused by initialising using MV? Easily fixed with smaller values for disallowed transitions.
% 
% Investigate class-wise prec and rec?

% what has happened to BSCC-seq?

% \begin{table*}
% % \small
% \begin{tabularx}{\textwidth}{| l | X | X | X | X | X | X | X | X | X |}
% \hline
% NER & \multicolumn{3}{|l|}{Span-level metrics}                     & \multicolumn{4}{|l|}{Token-level metrics} & & \\ \hline 
% & Prec. & Recall & F1 & F1 & AUC & CEE & $N_{invalid}$  & Notes & Hyperparams\\ \hline
% MV &  74.05 & 55.11 & 63.19 & 68.45 & .9406 & 7.72 & \MULTIPLY{.000883515461520577}{82494}{\sol}\ROUND[0]{\sol}{\sol}\sol &&\\
% MACE ! & 67.01 & 67.16 & 67.09 & 66.95 & .8385 & 2.87 & \MULTIPLY{.000837014647756336}{82494}{\sol}\ROUND[0]{\sol}{\sol}\sol & & .1, .1\\
% %DS & 77.23 & 7.94 & 73.95 & 75.50 & .9574 & 3.36 & \MULTIPLY{.000674387237803}{82494}{\sol}\ROUND[0]{\sol}{\sol}\sol \\ % max 10 iterations
% DS & 73.15 & 69.70 & 71.38 & 75.35 & .9548 & 4.10 & \MULTIPLY{.000651011392699372}{82494}{\sol}\ROUND[0]{\sol}{\sol}\sol & &\\
% IBCC & 73.92 & 69.56 & 71.67 & 75.41 & .9586 & .61 & \MULTIPLY{.00053475935828877}{82494}{\sol}\ROUND[0]{\sol}{\sol}\sol & & .1, 10, .1 \\ \hline
% HMM-Crowd* !& 77.67 & 7.05 & 73.67 & 75.33 & .9766 & 1.11 & 0 & Why better than BSCC-Vec-IF? Uses posteriors, different initialisation & 0, .1 \\ % 10 EM iterations
% HMM-Crowd-then-LSTM*! & 77.67 & 7.66 & 74.00 & 75.11 & .9058 & 13.92 & 0 & & 0, .1\\  \hline
% BSCC-seq-NF*! & 81.30 & 67.46 & 73.74 & 7.35 & .9585 & .45 & 0 & & 100, 100, 36\\ \hline
% BSCC-acc-IF & 77.60 & 52.88 & 62.90 & 67.42 & .9643 & 1.41 & 0 & & .1, 1, 1 \\
% BSCC-MACE-IF! & 44.42 & 79.92 & 57.10 & 57.93 & .9534 & 1.35 & 0 &  & .1, 1, .1\\
% BSCC-Vec-IF & 75.66 & 63.25 & 68.90 & 72.88 & .9747 & .93 & 0 &  &  .1, 10, .1\\
% %BSCC-Vec-IF & 81.01 & 62.72 & 7.71 & 71.28 & .9695 & .82 & 0 &  & 10, 1 \\
% % BSCC-Vec-IF & 81.01 & 63.16 & 71.01 & 71.38 & .9702 & .80 & 0 &  &  .1, 1, .1\\
% BSCC-IBCC-IF & 7.91 & 7.89 & 7.90 & 74.36 & .9739 & .87 & 0 &  & .1, .1, 1\\
% %BSCC-IBCC-IF & 75.43 & 74.21 & 74.82 & 74.33 & .9733 & .55 & 0 &  & 100, .1, 9\\
% BSCC-seq-IF & 
% 73.63 & 78.51 & 7.98 & 71.96 & .9015 & 1.45 & 0 & & .1, 10, 1\\ \hline
% %BSCC-seq-IF & 81.66 & 69.49 & 75.08 & 71.31 & .9707 & .44 & 0 & & 100, 36\\ \hline
% %BSCC-seq-IF & 81.88 & 7.66 & 75.86 & 73.94 & .9304 & .82 & 0 & & .1, 1, 1\\ \hline
% BSCC-IBCC-IF-then-LSTM*! & 71.53 & 69.88 & 7.70 & 7.61 & .9082 & 16.92 & 0 & Pre bug-fix result & 50, 9 \\% Post bug-fix (with BSCC-seq), the LSTM seems to have failed - sometimes more epochs are required? Compare to HMM-Crowd-then-LSTM & 50, 9\\
% %BSCC-IBCC+LSTM &\\ % don't need if we already know that including IF helps
% BSCC-seq-IF+LSTM*! & 81.67 & 69.64 & 75.17 & 71.32 & .9707 & .44 & 0 & & 100, 36 \\ 
% BCC-seq-IF+LSTM*! & & & & & & & & Running & 100, 36 \\ % no HMM because we integrate LSTM instead
% \hline
% \end{tabularx}
% \caption{Crowdsourced label aggregation performance on NER dataset: estimating true labels given crowdsourced labels.}
% \label{tab:aggregation_results_ner}
% \npnoround
% \end{table*}

\begin{table*}
% \small
\begin{tabularx}{\textwidth}{| l | Y | Y | Y | Y | Y | Y | Y | >{\raggedleft\arraybackslash}p{1.6cm} |}
\hline
NER & \multicolumn{3}{|l|}{Span-level metrics}                     & \multicolumn{4}{|l|}{Token-level metrics} & \\ \hline 
& Prec. & Recall & F1 & F1 & AUC & CEE & $N_{inval}$  & Hyper.\\ \hline
MV & 75.1 & 54.9 & 63.5 & %.796 & .581 & .672 & 
68.5 & .941 & 7.62 & 40 & \\ 
MACE & 71.2 & 64.9 & 67.9 & %.741 & .694 & .717 & 
71.6 & .837 & 1.14 & 35 & .1, .1 \\ 
DS & 74.9 & 69.7 & 72.2 & %.786 & .739 & .762 & 
75.4 & .955 & 4.16 & 25 &  \\ 
IBCC & 75.7 & 69.6 & 72.5 & %.791 & .735 & .762 & 
75.5 & .959 & \textbf{0.62} & 21 & .1, 10, .1 \\ 
\hline

HMM-crowd & 74.2 & 68.2 & 71.1 & %.784 & .717 & .749 & 
75.2 & \textbf{.977} & 1.08 & 1 & 0, .1 \\ 
HMM-crowd$\rightarrow$LSTM & 74.3 & 68.3 & 71.2 & %.785 & .717 & .749 & 
75.5 & .907 & 13.75 & 1 & 0, .1 \\ 
\hline

BSC-seq-notext & 75.8 & 65.4 & 70.2 & %.829 & .695 & .756 & 
70.8 & .894 & 1.15 & 0 & .1, 10, 1 \\ \hline

BSC-acc & 78.0 & 53.2 & 63.2 & 67.5 & .966 & 1.43 & 0 & .1, 1, 1 \\ 
BSC-MACE & 63.6 & 73.0 & 68.0 & %.562 & .803 & .661 & 
73.0 & .964 & 1.22 & 0 & .1, 10, .1 \\ 
BSC-CV & 76.8 & 62.7 & 69.0 & %.809 & .658 & .726 & 
72.8 & .974 & 1.05 & 0 & .1, 10, .1 \\ 
BSC-CM & 73.7 & 72.2 & 72.9 & 75.6 & .965 & 1.80 & 0 & .1, .1, .1 \\ 
BSC-seq & \textbf{77.0} & \textbf{72.0} & \textbf{74.4} & \textbf{76.4} & .913 & 1.35 & 0 & .1, 10, 1 \\ 
\hline

%BSC-seq$\rightarrow$LSTM & 76.6 & 70.9 & 73.6 & 75.6 & .906 & 12.71 & 0 & .1, 10, 1 \\ % old prior
BSC-seq$\rightarrow$c  LSTM & 76.2 & 70.2 & 73.1 & 75.8 & .902 & 12.98 & 0 & .1, 10, 1 \\ 
%BSC-seq+LSTM & 75.3 & 69.3 & 72.2 & 74.7 & .913 & 1.24 & 0 & .1, 10, 1 \\% old prior
BSC-seq+LSTM & 74.9 & 68.3 & 71.4 & 74.1 & .910 & 1.99 & 0 & .1, 10, 1 \\
BCC-seq+LSTM & 51.6 & 20.4 & 29.3 & 36.3 & .947 & 3.54 & 0 & .1, 10, 1 \\
\hline
\end{tabularx}
\caption{NER dataset: estimating true labels for documents that have been labelled by the crowd.}
\label{tab:aggregation_results_ner}
\npnoround
\end{table*}


% tuning tested the following hyperparameters on a subset of the validation data:
%diags = [.1, 1, 10, 50, 100]#[1, 50, 100]#[1, 5, 10, 50]
%factors = [.1, 1, 9, 36]

\begin{table*}
% \small
\begin{tabularx}{\textwidth}{| l | Y | Y | Y | Y | Y | Y | Y | >{\raggedleft\arraybackslash}p{1.6cm} |}
\hline
PICO & \multicolumn{3}{|l|}{Span-level metrics}                          & \multicolumn{4}{|l|}{Token-level metrics} &\\ \hline 
& Prec. & Recall & F1 & F1 & AUC & CEE & $N_{inval}$ & Hyper. \\ \hline
MV & 82.5 & 52.8 & 64.3 & 76.4 & .923 & 2.55 & 80 &  \\
MACE & 25.4 & 84.1 & 39.0 & 44.3 & .840 & 58.23 & 0 & .1, .1\\
DS & 71.3 & 66.3 & 68.7 & \textbf{79.3} & .934 & 0.44 & 54 &\\ 
IBCC & 72.1 & 66.0 & 68.9 & \textbf{79.3} & .935 & \textbf{0.27} & 37 & .1, 10, 10 \\ \hline

HMM-Crowd & 76.5 & 66.2 & 71.0 & 77.9 & \textbf{.944} & 0.79 & 0, .1 & \\ 
HMM-Crowd-then-LSTM & 76.5 & 66.5 & 71.2 & 78.2 & .868 & 12.94 & 0 & 0, .1 \\ \hline

BSCC-seq-notext & 81.2 & 59.2 & 68.5 & 59.8 & .922 & 0.73 & 0 & .1, .1, .1\\ \hline

BSCC-acc & \textbf{89.6} & 45.1 & 60.0 & 75.0 & .926 & 1.06 & 0 & .1, .1, 10 \\
BSCC-MACE & 46.7 & 84.4 & 60.1 & 68.5 & \textbf{.944} & 1.98 & 0 &  .1, 100, .1\\
BSCC-CV & 74.9 & 67.2 & 71.1 & 77.2 & .936 & 0.84 & 0 & .1, 1, .1\\
BSCC-CM & 61.8 & 76.3 & 68.3 & 74.8 & .939 & 1.37 & 0 & .1, 100, 1 \\
BSCC-seq & 74.7 & 73.6 & 74.2 & 58.8 & .929 & 0.97 & 0 & .1, .1, .1 \\ \hline 

BSCC-seq$\rightarrow$LSTM & 87.1 & 61.4 & 72.0 & 51.6 & .821 & 21.62 & 0 & .1, .1, .1 \\
BSCC-seq+LSTM & 75.1 & \textbf{77.3} & \textbf{76.2} & 51.9 & .934 & 0.68 & 0 & .1, .1, .1 \\
BCC-seq+LSTM & needs & to & be & rerun & & & & .1, .1, .1 \\
\hline
\end{tabularx}
\caption{PICO dataset: estimating true labels for documents that have been labelled by the crowd.}
\label{tab:aggregation_results_pico}
\end{table*}

\begin{figure*}
\centering
% \subfloat[BSCC-acc-IF]{
%   \includegraphics[width=0.9\textwidth, clip=True, trim=0 10 0 27]{figures/worker_models/acc}
% } \\
\subfloat[BSCC-Vec-IF]{
  \includegraphics[width=1\textwidth, clip=True, trim=0 10 0 27]{figures/worker_models/vec}
} \\
\subfloat[BSCC-MACE-IF]{
  \includegraphics[width=1\textwidth, clip=True, trim=0 10 0 27]{figures/worker_models/mace}
} \\
\subfloat[BSCC-IBCC-IF]{
  \includegraphics[width=1\textwidth, clip=True, trim=0 10 0 27]{figures/worker_models/ibcc}
} \\
\subfloat[BSCC-seq-IF, previous label = I]{
  \includegraphics[width=1\textwidth, clip=True, trim=0 10 0 27]{figures/worker_models/seq_prev0}
} \\
\subfloat[BSCC-seq-IF, previous label = O]{
  \includegraphics[width=1\textwidth, clip=True, trim=0 10 0 27]{figures/worker_models/seq_prev1}
} \\
\subfloat[BSCC-seq-IF, previous label = B]{
  \includegraphics[width=1\textwidth, clip=True, trim=0 10 0 27]{figures/worker_models/seq_prev2}
} \\
\caption{Confusion matrix representations from each BSCC-***-IF variant trained on the PICO datasets 
showing the different representations of workers. 
% Show how worker representation benefits from richer model: e.g. show differences between rows in IBCC compared to acc.
% Represent all types as IBCC-confusion matrix plots. seq will need more >= 3 plots! We can focus on PICO data to make it easier to view,
% or combiner NER classes into BIO. 
% If each row corresponds to one model, we have 7 rows (2 extra for seq). 
% Each row can then show either a selection of 5 workers, or we can cluster into 5 groups.
}
\label{fig:conf_mat_clusters}

\end{figure*}

\begin{figure}
\centering
\subfloat[prev. = I]{
  \includegraphics[width=.31\columnwidth, clip=True, trim=20 47 10 25]{figures/worker_models/seq+LSTM_prev0}
}
\subfloat[prev. = O]{
  \includegraphics[width=.31\columnwidth, clip=True, trim=20 47 10 25]{figures/worker_models/seq+LSTM_prev1}
}
\subfloat[prev. = B]{
  \includegraphics[width=.31\columnwidth, clip=True, trim=20 47 10 25]{figures/worker_models/seq+LSTM_prev2}
}
\caption{PICO dataset: confusion matrices inferred by BSC-Seq-IF+LSTM for the integrated LSTM. }
\label{fig:conf_mat_lstm}
\end{figure}

\subsubsection{Examples of Aggregation}

\begin{figure}
\centering
\subfloat[Example 1]{
  \includegraphics[width=1\columnwidth, clip=True, trim=20 47 10 25]{figures/placeholder}
}\\
\subfloat[Example 2]{
  \includegraphics[width=1\columnwidth, clip=True, trim=20 47 10 25]{figures/placeholder}
}\\
\subfloat[Example 3]{
  \includegraphics[width=1\columnwidth, clip=True, trim=20 47 10 25]{figures/placeholder}
}
\caption{Examples of different handling of annotator disagreement on PICO. 
Lines above the text show the crowd's annotations. Lines below show the aggregated annotations from MV, IBCC, HMM-Crowd and BSCC-seq-IF.
The sequential methods are able to resolve some issues, 
while non-sequential methods can lead to invalid annotations. }
\label{fig:disagreements}
\end{figure}

\subsection{Prediction using an LSTM Trained by the Crowd}\label{sec:task2}

In this task, we use the aggregation methods to train an LSTM sequence tagger \cite{lample2016}
to show whether integrating the LSTM with the aggregation method improves performance.
For the NER dataset, we train the aggregation methods on the $415$ crowd-labelled documents, as before,
then use the outputs to train the LSTM. We then evaluate the LSTM on the validation and test sets
in the original CoNLL dataset.
With the PICO dataset, we run the aggregators on the $3,649$ documents without gold labels, 
use the outputs to train the LSTM, then evaluate the LSTM on the validation and test splits from the gold-labelled data.


\begin{table*}
% \small
\begin{tabularx}{\textwidth}{| l | Y | Y | Y | Y | Y | Y | Y |}
\hline
NER & \multicolumn{3}{|l|}{Span-level metrics}                     & \multicolumn{4}{|l|}{Token-level metrics} \\ \hline 
& Prec. & Recall & F1 & F1 & AUC & CEE & $N_{inval}$  \\ \hline 
%HMM-Crowd-then-LSTM & 76.19 & 66.24 & 70.87 &\\ % original results from Nguyen 2017
%HMM-Crowd-then-LSTM & 72.75 & 68.26 & 70.43 & 35.73 & .8752 & 23.45 & 0 & V. close to Nguyen et al.~\shortcite{nguyen2017aggregating}\\ 
HMM-crowd$\rightarrow$LSTM & 68.8 & \textbf{65.4} & \textbf{67.1} & \textbf{67.5} & .910 & 15.21 & 1 \\ 
%LSTM & 83.19 & 57.12 & 67.73 \\ 
%LSTM-Crowd & 82.38 & 62.10 & 70.82 \\ \hline
%BSC-seq$\rightarrow$LSTM & \textbf{75.0} & \textbf{68.0} & \textbf{71.3} & \textbf{70.2} & .909 & 13.35 & 0 \\ % this was achieved before the priors were 'fixed' so that the disallowed count from banned transition from restricted_type1 -> restricted_type2 was transferred to unrestricted_type2 instead of restricted_type1. However, the method without LSTM improved...
BSC-seq$\rightarrow$LSTM & \textbf{71.5} & 59.3 & 64.9 & 66.7 & .867 & 17.43 & 0 \\  
%BSC-seq+LSTM & 71.7 & 62.7 & 66.9 & 65.9 & \textbf{.957} & \textbf{0.542} & 0 \\
BSC-seq+LSTM & 69.3 & 61.1 & 64.9 & 66.3 & \textbf{.945} & \textbf{0.89} & 0 \\  
BCC-seq+LSTM & 39.7 & 14.1 & 20.9 & 31.6 & .924 & 2.14 & 0 \\
\hline
\end{tabularx}
\caption{Prediction performance on NER test dataset with training on crowdsourced labels.}
\label{tab:prediction_results_ner}
\end{table*}

\begin{table*}
% \small
\begin{tabularx}{\textwidth}{| l | Y | Y | Y | Y | Y | Y | Y |}
\hline
PICO & \multicolumn{3}{|l|}{Span-level metrics (std.)}                          & \multicolumn{4}{|l|}{Token-level metrics (std.)} \\ \hline 
& Prec. & Recall & F1 & F1 & AUC & CEE & $N_{inval}$ \\ \hline
HMM-Crowd$\rightarrow$LSTM & \textbf{75.6} & 61.6 & 67.9 & \textbf{76.4} & .838 & 13.46 & 0\\ \hline
BSC-seq$\rightarrow$LSTM & 74.0 & \textbf{66.1} & \textbf{70.0} & 58.2 & .835 & 19.6 & 0 \\
BSC-seq+LSTM & 60.7 & 52.8 & 56.4 & 54.0 & \textbf{.899} & \textbf{0.48} & 0\\
BCC-seq+LSTM &\\
\hline
\end{tabularx}
\caption{Prediction performance on PICO test dataset with training on crowdsourced labels.}
\label{tab:prediction_results_pico}
\end{table*}


\section{Active Document Selection}

We run an active learning simulation to evaluate whether the proposed Bayesian approach and integrated LSTM
can improve the efficiency of the crowdsourcing process. 
The simulation is run separately for each method tested, and begins with the same initial set of randomly-chosen
documents taken from the same crowd-labelled sets used in Section \ref{sec:task2}.
We retrieve the crowdsourced labels for the selected documents, run the aggregation method,
then use its posterior probabilities to select a new batch of the $N_{batchsize}$ most uncertain documents that have not yet been labelled. 
We retrieve the annotations for the selected batch of documents, then repeat the process until
all of the availble crowd labels have been used.
We set $N_{batchsize}$ to one tenth of the crowd-labelled dataset size for each of the datasets. At each iteration,
we monitor progress by training an LSTM on the current output of the aggregation method, 
and testing its performance as in Section \ref{sec:task2}. 
With the NER dataset we also evaluate the output of aggregation method on the test set for the crowd-labelled documents. 
This is not possible with PICO datas because we do not have gold labels for documents labelled by the crowd.

The active learning process tested here employs \emph{uncertainty sampling}, which is a well-established heuristic \cite{active learning paper -- see tacl paper?}. The selection method and batch size could be fine-tuned for future applications -- the goal of our experiment in this paper was to test the benefits of the proposed aggregation methods,
rather than to establish a robust active learning approach.
 
\begin{figure*}
\centering
\subfloat[NER dataset]{
  \includegraphics[width=.7\columnwidth, clip=True, trim=20 47 10 25]{figures/placeholder}
}
\subfloat[PICO dataset]{
  \includegraphics[width=.7\columnwidth, clip=True, trim=20 47 10 25]{figures/placeholder}
}
\caption{Active learning simulation: prediction performance after each labelled batch is received. Mean scores over 10 repeats.}
\label{fig:al}
\end{figure*}

\section{Discussion}

The benefits of sequential models are more evident on the PICO dataset than on NER, which may be due to the longer sequences or the smaller number of labels, since PICO target classes are only B, I, or O, whereas the B and I tags for NER 
are compounded with PER, LOC, ORG or MISC tags. <show an example from each dataset, with our predictions from HMM, BAC...>

\section{Conclusions}

We proposed BSC, a novel Bayesian approach to aggregating sequence labels
that can be combined with several different models of annotator noise and bias.
To model the effect of dependencies between labels on annotator noise and bias, we introduced 
the \emph{seq} annotor model.
Our results demonstrated the benefits of BSC over established non-sequential methods, such 
as MACE, Dawid and Skene (DS), and IBCC.
%reinforce previous work that has demonstrated the benefits of modeling annotator reliability when aggregating noisy data, such as crowdsourced labels. 
We also showed the advantages of a Bayesian approach for active learning,
and that the combination of \emph{BSC} with \emph{seq} annotator model improves 
the state-of-the-art over HMM-crowd on three NLP tasks with different types of span annotations.
In future work, we plan to adapt active learning methods for easy deployment on crowdsourcing platforms,
and for investigate techniques for automatically selecting good hyperparameters without recourse to a development
set, which is often unavailable at the start of a crowdsourcing process.
%Its performance depends on the combination of sequential annotator model, label transition matrix, and 
%text model. 
%To enable more efficient training data collection,
% We further improved the quality of aggregated labels,
% by integrating existing 
% sequence taggers into our variational inference approach as black-box training and prediction functions.
% This technique performed well with larger amounts of labeled data, but may benefit from the use of pre-trained neural sequence taggers
% when the dataset is very small.
%We also found that including a simple conditional independence model of text features enables us to learn BSC-seq more effectively.
% Future work will evaluate integrating sequence taggers built on
% Bayesian deep learning, 
% which may improve active learning.
% We will also investigate
% %alternative data selection strategies to bootstrap active learning, 
% %and 
% how to set priors for %reduce over-confidence in predictions by including
% the reliability of black-box methods by testing them
% on other training sets of similar size.

%how to adapt hyper-parameters of the NN automatically in low-resource state?

% % In future, BSC-Seq could be applied to other sequential classification tasks beside span annotation.
% For example, the order of tasks that are intended to be exchangeable may affect the likelihood
% of the labels provided by the annotators\cite{mathur2017}. Seq-BCC could be applied to model the 
% propensity of the workers to choose certain labels given their previous labels, while the 
% ground truth sequence may be ignored.


%%%%%%%%%%%%%%%%%%%%%%%%%%%%%%%%%%%%%%%%%%%%%%%%%%%%%%%%%%%%%%%%%%%%%%%%%%%%%%%%

% use section* for acknowledgment
%\section*{Acknowledgments}

% \addcontentsline{toc}{chapter}{Bibliography}
%\bibliographystyle{apalike}
\bibliographystyle{IEEEtran}
\bibliography{simpson}

\end{document}
