\section{Conclusions}

We proposed BSC, a novel Bayesian approach to aggregating sequence labels
that can be combined with several different models of annotator noise and bias.
To model the effect of dependencies between labels on annotator noise and bias, we introduced 
the \emph{seq} annotator model.
Our results demonstrated the benefits of BSC over established non-sequential methods, such 
as MACE, Dawid and Skene (DS), and IBCC.
%reinforce previous work that has demonstrated the benefits of modeling annotator reliability when aggregating noisy data, such as crowdsourced labels. 
We also showed the advantages of a Bayesian approach for active learning,
and that the combination of \emph{BSC} with the \emph{seq} annotator model improves 
the state-of-the-art over HMM-crowd on three NLP tasks with different types of span annotations.

In future work, we plan to adapt active learning methods for easy deployment on crowdsourcing platforms,
and to investigate techniques for automatically selecting good hyperparameters without recourse to a development
set, which is often unavailable at the start of a crowdsourcing process.
%Its performance depends on the combination of sequential annotator model, label transition matrix, and 
%text model. 
%To enable more efficient training data collection,
% We further improved the quality of aggregated labels,
% by integrating existing 
% sequence taggers into our variational inference approach as black-box training and prediction functions.
% This technique performed well with larger amounts of labeled data, but may benefit from the use of pre-trained neural sequence taggers
% when the dataset is very small.
%We also found that including a simple conditional independence model of text features enables us to learn BSC-seq more effectively.
% Future work will evaluate integrating sequence taggers built on
% Bayesian deep learning, 
% which may improve active learning.
% We will also investigate
% %alternative data selection strategies to bootstrap active learning, 
% %and 
% how to set priors for %reduce over-confidence in predictions by including
% the reliability of black-box methods by testing them
% on other training sets of similar size.

%how to adapt hyper-parameters of the NN automatically in low-resource state?

% % In future, BSC-Seq could be applied to other sequential classification tasks beside span annotation.
% For example, the order of tasks that are intended to be exchangeable may affect the likelihood
% of the labels provided by the annotators\cite{mathur2017}. Seq-BCC could be applied to model the 
% propensity of the workers to choose certain labels given their previous labels, while the 
% ground truth sequence may be ignored.
